\theoremstyle{plain}
\newtheorem{teoElGrupoDeUnaCurva}{Teorema}[section]
\newtheorem{propoCocaracteresYParabolicosPropios}[teoElGrupoDeUnaCurva]%
	{Proposici\'{o}n}
\newtheorem{teoParabolicoSemidirecto}[teoElGrupoDeUnaCurva]{Proposici\'{o}n}

\theoremstyle{definition}
\newtheorem{defReductivo}[teoElGrupoDeUnaCurva]{Definici\'{o}n}
\newtheorem{ejemploCurvas}[teoElGrupoDeUnaCurva]{Ejemplo}

%-------------

A continuaci\'{o}n, veremos c\'{o}mo asociarle un esquema a un cocar\'{a}cter
del grupo $G$ y veremos que, en el caso de los grupos reductivos, esta
construcci\'{o}n da lugar a los subgrupos parab\'{o}licos de $G$.

Sean $X$ una variedad af\'{\i}n y $\cal O(X)$ su anillo de funciones regulares.
Sea $\lambda:\,\afin[1]\setmin\{0\}\rightarrow X$ un morfismo. En t\'{e}rminos
de morfismos de $k$-\'{a}lgebras, $\lambda=\pull\varphi$ para cierto morfismo
$\varphi:\,\cal O(X)\rightarrow k[T,T^{-1}]$. Decimos que \emph{existe el %
l\'{\i}mite de $\lambda(t)$ cuando $t$ tiende a $0$}, si $\lambda$ se extiende
a un morfismo $\tilde\lambda:\,\afin[1]\rightarrow X$.%
\footnote{
	\cite[\S~13.b]{MilneAlgebraicGroups}
}
En t\'{e}rminos de $\varphi$, pedimos que se correstrinja a
$\tilde\varphi:\,\cal O(X)\rightarrow k[T]$, es decir que
$\varphi(f)=f\circ\lambda$ sea un polinomio para toda $f\in\cal O(X)$.
Definimos, en ese caso,
\begin{align*}
	\lim_{t\to 0}\,\varphi(t) & \,:=\,\varphi(0)
	\text{ .}
\end{align*}
%

Sea, ahora, $\lambda:\,\multiplicativo\rightarrow G$ un cocar\'{a}cter (un
cocar\'{a}cter es como una curva en $G$). Entonces $\lambda$ determina una
acci\'{o}n en $G$ por conjugaci\'{o}n:
\begin{align*}
	t\cdot g & \,=\,\lambda(t)\,g\,\lambda(t)^{-1}
	\text{ .}
\end{align*}
%
Definimos un grupo algebraico $P(\lambda)$ asociado a esta acci\'{o}n por%
\footnote{
	\cite[Proposition~13.28]{MilneAlgebraicGroups}
}
\begin{align*}
	P(\lambda)(R) & \,:=\,\Big\{g\in G(R)\,:\,\text{existe }
		\lim_{t\to 0}\,t\cdot g\Big\}
	\text{ .}
\end{align*}
%

\begin{ejemploCurvas}\label{ejemplo:curvas:sl}
	\footnote{
		\cite[Example~13.31]{MilneAlgebraicGroups}
	}
	En $G=\SL[2]$, sea $\lambda$ el cocar\'{a}cter
	\begin{math}
		t\mapsto \diag(t,t^{-1})=
			\left[\begin{smallmatrix}
				t & \\
				& t^{-1}
			\end{smallmatrix}\right]
	\end{math}. La acci\'{o}n en $G$ est\'{a} dada por
	\begin{align*}
		\begin{bmatrix} t & \\ & t^{-1} \end{bmatrix}\,
		\begin{bmatrix} a & b \\ c & d \end{bmatrix}\,
		\begin{bmatrix} t & \\ & t^{-1} \end{bmatrix}^{-1} & \,=\,
		\begin{bmatrix}
			a & t^2 b \\ t^{-2}c & d
		\end{bmatrix}
	\end{align*}
	%
	y el l\'{\i}mite $t\to 0$ existe, si y s\'{o}lo si $c=0$. Entonces
	\begin{align*}
		P(\lambda) & \,=\,\bigg\{
			\begin{bmatrix} a & b \\ & a^{-1} \end{bmatrix}
			\bigg\}
		\text{ .}
	\end{align*}
	%
\end{ejemploCurvas}

\begin{ejemploCurvas}\label{ejemplo:curvas:gl}
	\footnote{
		\cite[Example~13.32]{MilneAlgebraicGroups}
	}
	En $G=\GL[3]$, $\lambda:\,t\mapsto\diag(t^{m_1},t^{m_2},t^{m_3})$ con
	$m_1\geq m_2\geq m_3$ fijos. Entonces
	\begin{align*}
		\text{si } m_1>m_2>m_3 \text{ ,} & \qquad
			P(\lambda) \,=\,\left\{
				\begin{bmatrix}
					* & * & * \\
					& * & * \\
					& & *
				\end{bmatrix}
				\right\} \text{ ,} \\
		\text{si } m_1=m_2>m_3 \text{ ,} & \qquad
			P(\lambda) \,=\,\left\{
				\begin{bmatrix}
					* & * & * \\
					* & * & * \\
					& & *
				\end{bmatrix}
				\right\}
			\text{ .}
	\end{align*}
	%
\end{ejemploCurvas}

\begin{defReductivo}\label{def:reductivo}
	\footnote{
		\cite[\S~6.46]{MilneAlgebraicGroups}
	}
	Dado un grupo algebraico suave $G$, su \emph{radical unipotente} es el
	subgrupo normal, suave, conexo y unipotente maximal; lo denotamos
	$R_u(G)$. Un grupo algebraico $G$ es \emph{reductivo}, si es suave,
	conexo y $R_u(G_{\algclos k})=e$.
\end{defReductivo}

\begin{teoElGrupoDeUnaCurva}\label{thm:elgrupodeunacurva}
	\footnote{
		\cite[Theorem~25.1]{MilneAlgebraicGroups}
	}
	Sea $G$ un grupo reductivo y sea
	$\lambda:\,\multiplicativo\rightarrow G$ un cocar\'{a}cter. Entonces
	$P(\lambda)$ es un grupo parab\'{o}lico y as\'{\i} son todos.
\end{teoElGrupoDeUnaCurva}

\begin{propoCocaracteresYParabolicosPropios}%
	\label{propo:cocaracteres:parabolicospropios}
	\footnote{
		\cite[Proposition~25.2]{MilneAlgebraicGroups}
	}
	Si $G$ es reductivo, entonces $G$ contiene subgrupos parab\'{o}licos
	propios, si y s\'{o}lo si posee un toro escindido no central.
\end{propoCocaracteresYParabolicosPropios}

\begin{proof}
	Demostramos \'{u}nicamente que la condici\'{o}n es necesaria. Si no hay
	toros escindidos no centrales en $G$, la imagen de cualquier
	cocar\'{a}cter $\lambda:\,\multiplicativo\rightarrow G$ debe caer en el
	centro $\lambda(\multiplicativo)\subset\centre(G)$ y, por lo tanto,
	$P(\lambda)=G$. Pero todos los subgrupos parab\'{o}licos son de la
	forma $P(\lambda)$ para alg\'{u}n cocar\'{a}cter.
\end{proof}

Para demostrar la suficiencia, tenemos que definir, dado un cocar\'{a}cter
$\lambda:\,\multiplicativo\rightarrow G$, el subgrupo%
\footnote{
	\cite[Proposition~13.29]{MilneAlgebraicGroups}
}
$U(\lambda)\leq P(\lambda)$ que a cada $k$-\'{a}lgebra $R$ le asigna
\begin{align*}
	U(\lambda)(R) & \,=\,\Big\{g\in G(R)\,:\,\lim_{t\to 0}\,t\cdot g=e
		\Big\}
	\text{ .}
\end{align*}
%
Necesitaremos, tambi\'{e}n, el centralizador
$Z(\lambda)=\centraliza[G](\lambda\,\multiplicativo)$.

\begin{ejemploCurvas}\label{ejemplo:unipotente:curvas:sl}
	Siguiendo con el Ejemplo~\ref{ejemplo:curvas:sl}, para el
	cocar\'{a}cter $\lambda(t)=\diag(t,t^{-1})$, el centralizador de
	$\lambda(\multiplicativo)$ est\'{a} dado por
	\begin{align*}
		Z(\lambda) & \,=\,\bigg\{
			\begin{bmatrix} a & \\ & a^{-1} \end{bmatrix}
			\bigg\}
		\text{ ,}
	\end{align*}
	%
	que es el toro maximal de $\SL[2]$. Si
	\begin{math}
		g=\left[\begin{smallmatrix}
			a & b \\ & d
		\end{smallmatrix}\right]\in P(\lambda)
	\end{math}, el l\'{\i}mite es
	\begin{math}
		\lim_{t\to 0}\,t\cdot g=
			\left[\begin{smallmatrix}
				a & \\ & d
			\end{smallmatrix}\right]
	\end{math}. De esto, deducimos que
	\begin{align*}
		U(\lambda) & \,=\,\bigg\{
			\begin{bmatrix} 1 & b \\ & 1 \end{bmatrix}
			\bigg\}
		\text{ .}
	\end{align*}
	%
\end{ejemploCurvas}

\begin{ejemploCurvas}\label{ejemplo:unipotente:curvas:gl}
	En el Ejemplo~\ref{ejemplo:curvas:gl}, $G=\GL[3]$ y
	$\lambda(t)=\diag(t^{m_1},t^{m_2},t^{m_3})$. La acci\'{o}n est\'{a}
	dada por
	\begin{align*}
		\lambda(t)\,
			\begin{bmatrix}
				a & b & c \\
				d & e & f \\
				g & h & i
			\end{bmatrix}\,\lambda(t)^{-1} & \,=\,
		\begin{bmatrix}
			a & t^{m_1-m_2}b & t^{m_1-m_3}c \\
			t^{m_2-m_1}d & e & t^{m_2-m_3}f \\
			t^{m_3-m_1}g & t^{m_3-m_2}h & i
		\end{bmatrix}
		\text{ .}
	\end{align*}
	%
	Si $m_1=m_2>m_3$,
	\begin{align*}
		Z(\lambda) \,=\,\left\{
			\begin{bmatrix}
				* & * & \\
				* & * & \\
				& & *
			\end{bmatrix}\right\} & \quad\text{y}\quad
		U(\lambda) \,=\,\left\{
			\begin{bmatrix}
				1 & 0 & * \\
				0 & 1 & * \\
				& & 1
			\end{bmatrix}\right\}
		\text{ .}
	\end{align*}
	%
\end{ejemploCurvas}

\begin{teoParabolicoSemidirecto}\label{thm:parabolicosemidirecto}
	\footnote{
		\cite[Theorem~13.33]{MilneAlgebraicGroups}. Este resultado
		tambi\'{e}n garantiza que $P$, $U$ y $Z$ son conexos, si $G$ lo
		es.
	}
	Si $G$ es un grupo algebraico suave, entonces $P(\lambda)$,
	$U(\lambda)$ y $Z(\lambda)$ son subgrupos algebraicos suaves de $G$.
	Adem\'{a}s, $U(\lambda)$ es normal en $P(\lambda)$ y unipotente%
	\footnote{
		\cite[Example~14.13]{MilneAlgebraicGroups}
	}
	y la multiplicaci\'{o}n induce un isomorfismo
	\begin{equation}
		\label{eq:parabolicosemidirecto}
		U(\lambda)\rtimes Z(\lambda)\,\rightarrow\,P(\lambda)
		\text{ .}
	\end{equation}
	%
\end{teoParabolicoSemidirecto}

Un grupo algebraico $G$ (arbitrario) se dice \emph{unipotente},%
\footnote{
	\cite[\S~6.45]{MilneAlgebraicGroups}
}
si toda representaci\'{o}n no nula posee un vector fijo (un autovector con
autovalor asociado $1$). En tal caso, si $(V,\rho)$ es una representaci\'{o}n
de dimensi\'{o}n $n<\infty$, existe una base de $V$ con respecto a la cual
$\rho(G)\subset\unipotentes[n]$ (el subgrupo de las matrices triangulares
superiores $\triangulares[n]$ con $1$ en la diagonal).

\begin{proof}[Fin de la demostraci\'{o}n de %
	\ref{propo:cocaracteres:parabolicospropios}]
	Si un grupo reductivo $G$ no posee subgrupos parab\'{o}licos propios,
	entonces, para todo cocar\'{a}cter $\lambda$, vale que $P(\lambda)=G$.
	Por otro lado, como $U(\lambda)$ es unipotente y normal en $G$
	reductivo, $U(\lambda)=e$. En particular,
	$\centraliza(\lambda\,\multiplicativo)=Z(\lambda)=G$ y
	$\lambda(\multiplicativo)\subset G$. Si
	$T\simeq\multiplicativo^k\leq G$ es un toro escindido de $G$, entonces
	existen cocaracteres
	$\lista{\lambda}{k}:\,\multiplicativo\rightarrow G$ tales que
	\begin{align*}
		T & \,=\,\lambda_1(\multiplicativo)\,\times\,\cdots\,\times\,
			\lambda_k(\multiplicativo) \,\subset\,\centre(G)
		\text{ .}
	\end{align*}
	%
	En definitiva, todo toro escindido es central.
\end{proof}
