\begin{teoParabolicosYEscindidos}\label{thm:parabolicosyescindidos}
	\footnote{
		C.f. \cite[Theorem~25.6 (b)]{MilneAlgebraicGroups}
	}
	Sea $P\leq G$ un subgrupo parab\'{o}lico y sea $S\leq P$ un toro
	escindido maximal \emph{de $P$}. Entonces $\centraliza[G](S)\leq P$
	(centralizador \emph{en $G$}).
\end{teoParabolicosYEscindidos}

De esto deducimos que, si $T\geq S$ es un toro que lo contiene, entonces
$T\leq P$. En particular, todo subgrupo parab\'{o}lico de $G$ contiene el
centralizador de un toro escindido maximal y los toros escindidos maximales de
un parab\'{o}lico son escididos maximales del grupo $G$. Podemos, entonces,
agrupar los subgrupos parab\'{o}licos en funci\'{o}n de si contienen o no al
centralizador de un toro escindido maximal:
\begin{align*}
	\Big\{P\leq G\text{ parab. (min.)}\Big\} & \,=\,
		\bigcup_{
			\begin{smallmatrix}
				S\leq G \\[1pt]
				\text{esc. max.}
			\end{smallmatrix}}\,
			\Big\{ P\leq G\text{ parab. (min.)}\,:\,
				P\geq\centraliza[G](S)\Big\}
	\text{ .}
\end{align*}
%

Expresado de otra manera, an\'{a}logamente a lo que sucede en el caso
algebraicamente cerrado entre subgrupos de Borel y toros maximales, todo
subgrupo parab\'{o}lico minimal $P\leq G$ forma parte de un par $(P,S)$ donde
$S$ es un toro escindido maximal contenido en $P$.

\begin{teoParabolicosConjugados}\label{thm:parabolicosconjugados}
	\footnote{
		\cite[Theorem~25.8]{MilneAlgebraicGroups}
	}
	Dos subgrupos parab\'{o}licos minimales de $G$ son conjugados por un
	elemento de $G(k)$.
\end{teoParabolicosConjugados}

\begin{teoEscindidosConjugados}\label{thm:escindidosconjugados}
	\footnote{
		\cite[Theorem~25.10]{MilneAlgebraicGroups}
	}
	Dos toros escindidos maximales de $G$ son conjugados por un elemento de
	$G(k)$.
\end{teoEscindidosConjugados}

Sea $S\leq G$ un toro escindido maximal de $G$. Sea $P_0$ un subgrupo
parab\'{o}lico minimal (alguno existe) y sea $S_0\leq G$ un toro escindido
maximal de $G$ tal que $S_0\leq P_0$. Por el Teorema~%
\ref{thm:escindidosconjugados}, existe $g\in G(k)$ tal que $S=gS_0g^{-1}$.
Entonces $S$ est\'{a} contenido en el subgrupo parab\'{o}lico minimal
$gP_0g^{-1}$.%
\footnote{
	Para ver que es minimal, recurrimos al Teorema~%
	\ref{thm:parabolicosconjugados}. Si $P_0'\leq gP_0g^{-1}$ es
	parab\'{o}lico minimal, existe $h\in G(k)$ tal que $hP_0h^{-1}=P_0'$.
	En particular, $(g^{-1}h)P_0(g^{-1}h)^{-1}$ es parab\'{o}lico y
	est\'{a} contenido en $P_0$. Por minimalidad de $P_0$, deben ser
	iguales y, as\'{\i}, $P_0'=gP_0g^{-1}$, tambi\'{e}n.
}
En definitiva, todo toro escindido maximal $S$ forma parte de un par $(P,S)$
donde $P\leq G$ es un subgrupo parab\'{o}lico minimal que lo contiene.

Tambi\'{e}n son v\'{a}lidos los an\'{a}logos de \ref{coro:normalizadordeltoro}
y de \ref{thm:paresconjugados}. Se demuestran de manera similar.

\begin{coroNormalizadorDelEscindido}\label{coro:normalizadordelescindido}
	Sea $S\leq G$ un toro escindido maximal y sea $P\geq S$ un subgrupo
	parab\'{o}lico minimal que lo contiene. El normalizador
	$\normaliza[G](S)$ act\'{u}a transitivamente en el conjunto de
	conjugados de $P$ que contienen a $S$.
\end{coroNormalizadorDelEscindido}

\begin{teoParesParabolicoEscindidoConjugados}%
	\label{thm:paresparabolicoescindidoconjugados}
	Dos pares $(P,S)$ y $(P',S')$ conformados por subgrupos parab\'{o}licos
	minimales $P,P'$ y toros escindidos maximales $S,S'$ tales que
	$P\geq S$ y $P'\geq S'$ son conjugados por un elemento de $G(k)$.
\end{teoParesParabolicoEscindidoConjugados}

