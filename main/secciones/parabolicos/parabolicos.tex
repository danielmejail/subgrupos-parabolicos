\theoremstyle{plain}
\newtheorem{teoLieKolchin}{Teorema}[section]
\newtheorem{propoTriangulares}[teoLieKolchin]{Proposici\'{o}n}
\newtheorem{teoBorelConjugados}[teoLieKolchin]{Teorema}
\newtheorem{teoTorosConjugados}[teoLieKolchin]{Teorema}
\newtheorem{coroNormalizadorDelToro}[teoLieKolchin]{Corolario}
\newtheorem{teoParesConjugados}[teoLieKolchin]{Teorema}
\newtheorem{propoParabolicosPropios}[teoLieKolchin]{Proposici\'{o}n}

\theoremstyle{definition}
\newtheorem{defBorel}[teoLieKolchin]{Definition}
\newtheorem{obsBorel}[teoLieKolchin]{Observaci\'{o}n}
\newtheorem{ejemploTriangulares}[teoLieKolchin]{Ejemplo}
\newtheorem{defParabolico}[teoLieKolchin]{Definici\'{o}n}
\newtheorem{obsParabolicosPropios}[teoLieKolchin]{Observaci\'{o}n}
\newtheorem{defBorelYParabolicos}[teoLieKolchin]{Definiciones}
\newtheorem{obsBorelYParabolicos}[teoLieKolchin]{Observaci\'{o}n}
\newtheorem{defQuasiSplit}[teoLieKolchin]{Definici\'{o}n}

%-------------

\subsection{en grupos algebraicos suaves}\label{subsec:parabolicos:variedades}
En esta secci\'{o}n, $G$ denota un grupo algebraico conexo sobre un cuerpo
algebraicamente cerrado $k=\algclos k$. Asumimos, tambi\'{e}n, que $G$ es
geom\'{e}tricamente reducido, o, lo que es lo mismo, un grupo algebraico suave,
es decir, una variedad.%
\footnote{
	\cite[pp.~12,13]{MilneAlgebraicGroups}
}

\begin{defBorel}\label{def:borel}
	\footnote{
		\cite[Definition~17.6]{MilneAlgebraicGroups}
	}
	Un \emph{subgrupo de Borel} de $G$ es un subgrupo $B\leq G$ soluble,
	conexo y maximal con respecto a estas propiedades.
\end{defBorel}

\begin{obsBorel}\label{obs:def:borel}
	En esta secci\'{o}n --en particular, en la Definici\'{o}n~%
	\ref{def:borel}-- ``subgrupo'' quiere decir subgrupo algebraico que
	es subvariedad. En particular, un subgrupo de Borel $B\leq G$ es un
	grupo algebraico \emph{suave}.%
	\footnote{
		El lunes hab\'{\i}a dicho, err\'{o}neamente, que no le
		ped\'{\i}amos a $B$ que fuese suave.
	}
\end{obsBorel}

Un grupo abstracto $G$ es \emph{soluble}, si existe una cadena finita de
subgrupos
\begin{align*}
	& G \,=\, G_0\,\supset\,G_1\,\supset\,\cdots\,\supset\,G_s\,=\,e
\end{align*}
%
tal que $G_{i+1}\triangleleft G_i$ y que cada cociente $G_i/G_{i+1}$ sea
abeliano.

\begin{ejemploTriangulares}\label{ejemplo:triangulares}
	En $G=\GL[n]$, el subgrupo de matrices triangulares superiores
	$B=\triangulares[n]$ es un subgrupo de Borel. Por ejemplo, para $n=4$,
	\begin{align*}
		\triangulares[4] \,=\,
			\left[\begin{smallmatrix}
				* & * & * & * \\
				& * & * & * \\
				& & * & * \\
				& & & *
			\end{smallmatrix}\right] & \,\supset\,
			\mathsf{U}_4\,=\,\left[\begin{smallmatrix}
				1 & * & * & * \\
				& 1 & * & * \\
				& & 1 & * \\
				& & & 1
			\end{smallmatrix}\right] \,\supset\,
			\left[\begin{smallmatrix}
				1 & 0 & * & * \\
				& 1 & 0 & * \\
				& & 1 & 0 \\
				& & & 1
			\end{smallmatrix}\right] \,\supset\,
			\left[\begin{smallmatrix}
				1 & 0 & 0 & * \\
				& 1 & 0 & 0 \\
				& & 1 & 0 \\
				& & & 1
			\end{smallmatrix}\right] \,\supset\,
			\left[\begin{smallmatrix}
				1 & 0 & 0 & 0 \\
				& 1 & 0 & 0 \\
				& & 1 & 0 \\
				& & & 1
			\end{smallmatrix}\right]
	\end{align*}
	%
	Esta sucesi\'{o}n muestra, tambi\'{e}n, que $\unipotentes[4]$ es
	\emph{nilpotente}.
\end{ejemploTriangulares}

Siguiendo con el Ejemplo~\ref{ejemplo:triangulares},

\begin{propoTriangulares}\label{propo:triangulares}
	todo subgrupo de Borel de $\GL[n]$ es conjugado a $\triangulares[n]$.
\end{propoTriangulares}

Un grupo algebraico $G$ (arbitrario) se dice \emph{triagonalizable},%
\footnote{
	\cite[Definition~16.1]{MilneAlgebraicGroups}
}
si toda representaci\'{o}n no nula posee un autovector. En tal caso, si
$(V,\rho)$ es una representaci\'{o}n de dimensi\'{o}n $n<\infty$, existe una
base de $V$ con respecto a la cual $\rho(G)\subset\triangulares[n]$. La
demostraci\'{o}n de la Proposici\'{o}n~\ref{propo:triangulares} se basa en el
siguiente resultado.

\begin{teoLieKolchin}[Lie-Kolchin]\label{thm:liekolchin}
	\footnote{
		\cite[Theorem~16.30]{MilneAlgebraicGroups}
	}
	Sobre un cuerpo algebraicamente cerrado, un grupo algebraico suave y
	conexo es soluble, si y s\'{o}lo si es triagonalizable.
\end{teoLieKolchin}

\begin{proof}[Demostraci\'{o}n de \ref{propo:triangulares}]
	Sea $B\leq\GL[n]$ un subgrupo de Borel. De acuerdo con la
	Observaci\'{o}n~\ref{obs:def:borel}, como $B$ es suave, conexo y
	soluble, podemos aplicar el Teorema~\ref{thm:liekolchin}. Concluimos,
	entonces, que la representaci\'{o}n $B\hookrightarrow\GL[k^n]$ admite
	una base (de $k^n$) con respecto a la cual $B\subset\triangulares[n]$.
	Dicho de otra manera, existe $g\in\GL[n](k)$ tal que
	\begin{align*}
		gBg^{-1} & \,\subset\,\triangulares[n]
		\text{ .}
	\end{align*}
	%
	Finalmente, por maximalidad de $B$, la inclusi\'{o}n anterior es una
	igualdad.
\end{proof}

Podemos expresar la conclusi\'{o}n de la Proposici\'{o}n~%
\ref{propo:triangulares} de una manera m\'{a}s invariante usando la noci\'{o}n
de \emph{banderas}.

Sea $V$ un $k$-espacio vectorial. Una \emph{bandera} en $V$ es una cadena
finita
\begin{equation}
	\label{eq:bandera}
	F \quad:\quad 0\,=\,V_0\,\subsetneq\,V_1\,\subsetneq\,\cdots\,
		\subsetneq\,V_s\,=\,V
\end{equation}
%
de subespacios de $V$. El \emph{estabilizador de una bandera $F$} como en
\eqref{eq:bandera} es el subgrupo $B_F\leq\GL[V]$ m\'{a}s grande con la
propiedad de que, para todo $i$
\begin{align*}
	B_F\,\cdot\,V_i & \,\subset\,V_i
	\text{ .}
\end{align*}
%
Decimos que una bandera como \eqref{eq:bandera} es \emph{maximal}, si
$s=\dim\,V$; en tal caso, la codimensi\'{o}n de $V_{i-1}$ en $V_i$ es $1$.

Sea $F$ una bandera maximal en $V$ y sea $B_F$ su estabilizador. Entonces,
tomando como base de $V$ un conjunto de la forma $\{\lista{v}{s}\}$ tal que
$v_i\in V_i\setmin V_{i-1}$, vemos que $B_F$ coincide con el grupo
$\triangulares[n]$. Rec\'{\i}procamente, si $B\leq\GL[V]$ es un subgrupo de
Borel, como $B$ es triagonalizable, existe $v_1\in V$ tal que
$V_1:=\generado{v_1}$ es un autoespacio para la acci\'{o}n de $B$. Esta
acci\'{o}n desciende al cociente $V/V_1$ y concluimos que existe
$v_2\in V\setmin V_1$ tal que $V_2:=\generado{v_1,v_2}$ sea un autoespacio para
$B$. Inductuvamente, por la ``triagonalizabilidad'' de $B$, podemos construir
una bandera maximal $F=(V_i)_i$ tal que $B\cdot V_i\subset V_i$. En particular,
$B\subset B_F$ y, por maximalidad de $B$ en tanto subgrupo soluble conexo,
\begin{align*}
	B & \,=\, B_F
	\text{ .}
\end{align*}
%
Es decir, los subgrupos de Borel de $\GL[V]$ son, exactamente, los
estabilizadores de banderas en $V$ maximales. Por \'{u}ltimo, como $\GL[V]$
act\'{u}a transitivamente en el conjunto de bases de $V$, concluimos que todos
los subgrupos de Borel de $\GL[V]$ son conjugados (por un elemento de
$\GL[V](k)$).

Esta propiedad del grupo general lineal es v\'{a}lida en general.

\begin{teoBorelConjugados}\label{thm:borelconjugados}
	\footnote{
		\cite[Theorem~17.9]{MilneAlgebraicGroups}
	}
	Dos subgrupos de Borel de $G$ son conjugados por un elemento de $G(k)$.
\end{teoBorelConjugados}

\begin{obsBorel}\label{obs:borel:existencia}
	Existen subgrupos de Borel. Podemos empezar con cualquier subgrupo
	conexo soluble (hay, al menos, uno: $\{e\}$) y considerar todos los
	subgrupo conexos solubles que lo contienen. Los de dimensi\'{o}n
	m\'{a}xima son los subgrupos de Borel.%
	\footnote{
		La \emph{dimensi\'{o}n} de un esquema irreducible de tipo
		finito se define como el supremo (m\'{a}ximo) de las longitudes
		de cadenas
		\begin{align*}
			Z & \,=\,Z_d\,\subset\,\cdots\,Z_1\,\subset\,Z_0
		\end{align*}
		%
		de subesquemas cerrados \cite[\S~A.24]{MilneAlgebraicGroups}.
		Por otro lado, los subgrupos algebraicos de un grupo algebraico
		son cerrados para la topolog\'{\i}a Zariski
		\cite[Proposition~1.41]{MilneAlgebraicGroups}. En particular,
		si $H\leq H'\leq G$ son subgrupos, entonces o bien $H=H'$, o
		bien $\dim H<\dim{H'}$.
	}
\end{obsBorel}

Si $T\leq G$ es un toro, entonces es conexo y
soluble y est\'{a} contenido en un subgrupo soluble y conexo maximal, es decir,
en un subgrupo de Borel. A un par $(B,T)$ donde $T\leq G$ es un toro maximal y
$B\leq G$ es un subgrupo de Borel que lo contiene se lo llama \emph{par de %
Borel}. As\'{\i}, todo toro maximal forma parte de un par de Borel y, como hay
al menos, un par de Borel y los subgrupos de Borel son todos conjugados, todo
subgrupo de Borel forma parte de un par de Borel.

\begin{teoTorosConjugados}\label{thm:torosconjugados}
	\footnote{
		\cite[Theorem~17.10]{MilneAlgebraicGroups}
	}
	Dos toros maximales de $G$ son conjugados por un elemento de $G(k)$.
\end{teoTorosConjugados}

% \begin{coroNormalizadorDelToro}\label{coro:normalizadordeltoro}
	% \footnote{
		% \cite[Corollary~17.11]{MilneAlgebraicGroups}
	% }
	% Sea $T\leq G$ un toro maximal y sea $B\geq T$ un subgrupo de Borel que
	% lo contiene. El normalizador $\normaliza[G](T)$ act\'{u}a
	% transitivamente en el conjunto de conjugados de $B$ que contienen a
	% $T$.
% \end{coroNormalizadorDelToro}
% 
% \begin{proof}
	% Si $g\in G(k)$ y $gBg^{-1}\supset T$, entonces
	% $T,gTg^{-1}\subset gBg^{-1}$. Pero $h(gTg^{-1})h^{-1}=T$ para cierto
	% $h\in gBg^{-1}$ (tomando $G=gBg^{-1}$ en el Teorema~%
	% \ref{thm:torosconjugados}). En particular, $hg\in\normaliza[G](T)$ y
	% $(hg)B(hg)^{-1}=gBg^{-1}$.
% \end{proof}
% 
% \begin{teoParesConjugados}\label{thm:paresconjugados}
	% \footnote{
		% \cite[Proposition~17.13]{MilneAlgebraicGroups}
	% }
	% Dos pares de Borel de $G$ son conjugados por un elemento de $G(k)$.
% \end{teoParesConjugados}
% 
% \begin{proof}
	% Sean $(B,T)$ y $(B',T')$ dos pares de Borel. Existe $g\in G(k)$ tal que
	% $gTg^{-1}\subset gB'g^{-1}=B$ y existe $h\in B(k)$ tal que
	% $h(gT'g^{-1})h^{-1}=T$. En particular, $(hg)B'(hg)^{-1}=B$,
	% tambi\'{e}n.
% \end{proof}
% 
\begin{defParabolico}\label{def:parabolico}
	\footnote{
		C.f. \cite[Theorem~17.16]{MilneAlgebraicGroups}
	}
	Un \emph{subgrupo parab\'{o}lico} de $G$ es un subgrupo $P\leq G$ que
	contiene alg\'{u}n subgrupo de Borel.
\end{defParabolico}

El grupo $G$ es, trivialmente, parab\'{o}lico.

\begin{propoParabolicosPropios}\label{propo:variedades:parabolicospropios}
	\footnote{
		\cite[Corollary~17.17]{MilneAlgebraicGroups}
	}
	El grupo $G$ contiene subgrupos parab\'{o}licos propios, si y s\'{o}lo
	si no es soluble.
\end{propoParabolicosPropios}

\begin{proof}
	Si $G$ es soluble, entonces $B=G$ es el \'{u}nico subgrupo de Borel y,
	en particular $P=G$ es el \'{u}nico subgrupo parab\'{o}lico.
	Rec\'{\i}procamente, si $G$ no es soluble y $B\leq G$ es un subgrupo de
	Borel, necesariamente, $B\not =G$. El subgrupo $B$ es un subgrupo
	parab\'{o}lico propio.
\end{proof}

\begin{obsParabolicosPropios}
	En esta demostraci\'{o}n usamos que siempre existen subgrupos de Borel
	si $G$ es algebraico y suave, definido sobre un cuerpo algebraicamente
	cerrado.
\end{obsParabolicosPropios}

\begin{ejemploTriangulares}\label{ejemplo:triangulares:parabolicos}
	En $G=\GL[4]$, los subgrupos
	\begin{align*}
		\left\{ \begin{bmatrix}
				* & * & * & * \\
				* & * & * & * \\
				& & * & * \\
				& & * & *
			\end{bmatrix} \right\}
		& \quad\text{y}\quad
		\left\{ \begin{bmatrix}
				* & * & * & * \\
				& * & * & * \\
				& * & * & * \\
				& & & *
			\end{bmatrix} \right\}
	\end{align*}
	%
	contiene a $\triangulares[4]$ y son parab\'{o}licos. Un poco m\'{a}s en
	general, en $\GL[n]$, los subgrupos de matrices triangulares superiores
	en bloques (para cualquier elecci\'{o}n de tama\~{n}o de los bloques
	diagonales) contiene a $\triangulares[n]$ y son parab\'{o}licos.%
	\footnote{
		Ver el Ejemplo~\ref{ejemplo:}.
	}
\end{ejemploTriangulares}

La descripci\'{o}n de los subgrupos parab\'{o}licos del Ejemplo~%
\ref{ejemplo:triangulares:parabolicos} se puede dar en t\'{e}rminos de
banderas. Dados un $k$-espacio vectorial $V$ y banderas $F=(V_i)_i$ y
$\widetilde F=(\widetilde V_j)_j$ en $V$, decimos que \emph{$\widetilde F$ %
extiende a $F$}, si, para cada \'{\i}ndice $i$, existe $j=j(i)$ tal que
$V_i=\widetilde V_j$. Si llamamos $P_F$ al estabilizador de $F$ y
$B_{\widetilde F}$ al estabilizador de $\widetilde F$, entonces $P_F$ es
parab\'{o}lico, porque $P_F\geq B_{\widetilde F}$ y $B_{\widetilde F}$ es de
Borel.%
\footnote{
	Esto no significa que todos los subgrupos parab\'{o}licos de $\GL[n]$
	sean de esta forma, es decir, conjugados a un subgrupo de matrices
	triangulares superiores en bloques. Que esto es as\'{\i} es
	consecuencia de la expresi\'{o}n \eqref{eq:}.
}

\begin{obsBorel}\label{obs:borel:parabolicos}
	Los subgrupos de Borel son los subgrupos parab\'{o}licos minimales.
\end{obsBorel}


\subsection{en grupos algebraicos}\label{subsec:parabolicos:algebraicos}
En esta secci\'{o}n, $G$ denota un grupo algebraico, sin otras imposiciones.
Dejamos, tambi\'{e}n, de asumir que $k=\algclos k$.

\begin{defBorelYParabolicos}\label{def:borelyparabolicos}
	\footnote{
		\cite[Definition~1.29]{GetzHahnAutomorphicRepresentations}.
		C.f. \cite[\S~17.66]{MilneAlgebraicGroups}
	}
	Un subgrupo $B\leq G$ es un \emph{subgrupo de Borel}, si
	$B_{\algclos k}$ es un subgrupo de Borel de $G_{\algclos k}$ en el
	sentido de la Definici\'{o}n~\ref{def:borel}. Un subgrupo $P\leq G$ es
	un \emph{subgrupo parab\'{o}lico}, si $P_{\algclos k}$ es un subgrupo
	parab\'{o}lico de $G_{\algclos k}$ en el sentido de la Definici\'{o}n~%
	\ref{def:parabolico}.
\end{defBorelYParabolicos}

\begin{obsBorelYParabolicos}\label{obs:borelyparabolicos:borel}
	Puede suceder que un grupo $G$ no posea subgrupos de Borel. En la
	definici\'{o}n, no estamos pidiendo, sobre $k$, las mismas condiciones
	que antes ped\'{\i}amos sobre la clausura algebraica. Espec\'{\i}%
	ficamente, un subgrupo soluble, conexo maximal en $G$ \emph{no %
	necesariamente es un subgrupo de Borel}; no podemos repetir el
	argumento de la Observaci\'{o}n~\ref{obs:borel:existencia}.
\end{obsBorelYParabolicos}

\begin{obsBorelYParabolicos}\label{obs:borelyparabolicos:parabolicos}
	Por la misma raz\'{o}n, podr\'{\i}a darse el caso de un grupo que
	no posea subgrupos parab\'{o}licos propios (el grupo mismo
	\emph{siempre} es parab\'{o}lico). Tambi\'{e}n podr\'{\i}a ocurrir que
	existan subgrupos parab\'{o}licos, pero que \'{e}stos no contengan
	subgrupos de Borel.
\end{obsBorelYParabolicos}

\begin{defQuasiSplit}\label{def:quasisplit}
	Si $G$ posee un subgrupo de Borel, se dice que $G$ es \emph{quasi-%
	split}. Los subgrupos de Borel, si existieren, ser\'{a}n subgrupos
	parab\'{o}licos minimales.
\end{defQuasiSplit}

