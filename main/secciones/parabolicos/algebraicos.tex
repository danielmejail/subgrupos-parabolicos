En esta secci\'{o}n, $G$ denota un grupo algebraico, sin otras imposiciones.
Dejamos, tambi\'{e}n, de asumir que $k=\algclos k$.

\begin{defBorelYParabolicos}\label{def:borelyparabolicos}
	\footnote{
		\cite[Definition~1.29]{GetzHahnAutomorphicRepresentations}.
		C.f. \cite[\S~17.66]{MilneAlgebraicGroups}
	}
	Un subgrupo $B\leq G$ es un \emph{subgrupo de Borel}, si
	$B_{\algclos k}$ es un subgrupo de Borel de $G_{\algclos k}$ en el
	sentido de la Definici\'{o}n~\ref{def:borel}. Un subgrupo $P\leq G$ es
	un \emph{subgrupo parab\'{o}lico}, si $P_{\algclos k}$ es un subgrupo
	parab\'{o}lico de $G_{\algclos k}$ en el sentido de la Definici\'{o}n~%
	\ref{def:parabolico}.
\end{defBorelYParabolicos}

\begin{obsBorelYParabolicos}\label{obs:borelyparabolicos:borel}
	Puede suceder que un grupo $G$ no posea subgrupos de Borel. En la
	definici\'{o}n, no estamos pidiendo, sobre $k$, las mismas condiciones
	que antes ped\'{\i}amos sobre la clausura algebraica. Espec\'{\i}%
	ficamente, un subgrupo soluble, conexo maximal en $G$ \emph{no %
	necesariamente es un subgrupo de Borel}; no podemos repetir el
	argumento de la Observaci\'{o}n~\ref{obs:borel:existencia}.
\end{obsBorelYParabolicos}

\begin{obsBorelYParabolicos}\label{obs:borelyparabolicos:parabolicos}
	Por la misma raz\'{o}n, podr\'{\i}a darse el caso de un grupo que
	no posea subgrupos parab\'{o}licos propios (el grupo mismo
	\emph{siempre} es parab\'{o}lico). Tambi\'{e}n podr\'{\i}a ocurrir que
	existan subgrupos parab\'{o}licos, pero que \'{e}stos no contengan
	subgrupos de Borel.
\end{obsBorelYParabolicos}

\begin{defQuasiSplit}\label{def:quasisplit}
	Si $G$ posee un subgrupo de Borel, se dice que $G$ es \emph{quasi-%
	split}. Los subgrupos de Borel, si existieren, ser\'{a}n subgrupos
	parab\'{o}licos minimales.
\end{defQuasiSplit}
